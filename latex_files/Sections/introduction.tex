%!TEX root = ../main.tex

\documentclass[../main.tex]{subfiles}
\begin{document}

\section{Introduction}
\label{sec:introduction}
The economic literature of recent years has identified significant changes in demand on the labor markets in Europe and the US, particularly during the 1980s and 90s. The concept of a skill-biased technological change resulting in higher returns to skills is well established and widely recognized to be a driver behind these changes in labor demand. It builds on the idea that developments in production technology increase the demand for more skilled workers (\cite{acemoglu2011skills}). This increase in returns to skills, in particular returns to education, has been accompanied by shifts in the composition of employment. These shifts often are summarized under the term \textit{polarization} and are characterized by a simultaneous growth in both, high skill - high paying and low skill - low paying occupations, leading to relative decline in medium skill - medium paying occupations. An overview of these labor market trends is presented, e.g., in Acemoglu and Autor (\citeyear{acemoglu2011skills}).
% TODO: Finde einen Aufhänger zur Motivation des Themas, beispiesweise Bericht über die wachsenden Unterschiede im Gehaltswachstum nach Bidungsebene.
\\
The most widely discussed source of this polarization is the "routinization hypothesis", introduced by Autor et al. (\citeyear{autor2003skill}). This approach on skill-biased technological change is based on the idea of routine tasks being easier to either automate or offshore, resulting in a technology driven decrease in demand for workers executing such tasks. In this framework, the emerging computer technology generally complements high skilled workers who tend to execute non-routine tasks while substituting for low skilled workers who are more likely to be involved in routine tasks. These two mechanisms lead to an increase in relative demand for workers that have a comparative advantage in the execution of non-routine tasks and thus are favorable for relatively high educated workers (\cite{autor2003skill}). 
\\
At the opposite end of the skill spectrum, the increase in service occupation employment is accountable for the relative growth in employment in low skilled and low payed occupations (\cite{autor2010inequality}). Building on this, Goos and Manning (\citeyear{goos2007lousy}) present evidence that both, high and low skill occupations are likely to involve non-routine tasks, while medium skilled occupations appear to consist of more routine tasks. By examining the tasks that typically are executed in the context of an occupation, the authors can thus show that the "routinization hypothesis" is consistent with the observed polarization of labor demand. A comprehensive review of related literature is presented in, e.g., Green and Sand (\citeyear{green2015has}).
\\
% TODO: Maybe make clearer that there is a break between 1980/90s and 2000s/later: Chnages in labor demand of ealier years are explained quite well by the "simpler" approaches - but changes in later year cannot be explained as good. Thus the need for a finer analysis on task basis.
Studies focusing on more recent years reveal, in contrast to the polarization hypothesis, little to no employment growth of high skilled occupations (see, e.g., \cite{beaudry2016great}). Analyzing relative employment data, Deming (\citeyear{deming2017growing}) presents evidence of differences in the growth rates of employment in various high skilled and cognitive occupations. Within this group of occupations, he shows that those which are in the the fields of science, technology, engineering and mathematics (STEM) experienced a relative decline in employment. At the same time, managers, teachers, nurses as well as therapists, physicians and economists are identified as fastest growing occupations within the cognitive category. This motivates the conjectured connection that it is not the general skill level required in execution of an occupation, but especially the content of interpersonal interactions, which determines the occupational growth.
\\
Building on this, Deming (\citeyear{deming2017growing}) provides evidence on the importance of social skill on labor market outcomes. Further, he hypothesizes that one potential driver of this occupational growth in favor of interpersonal interaction intensive occupations might be that this task shows to be hard to automate. Autor (\citeyear{autor2015there}) argues that generally those skills and tasks that cannot be substituted by automation are complemented by it. 
\\
Thus, a  more nuanced view on the relative importance of different tasks contained within occupations might be helpful in understanding changes in growth of occupational employment. Instead of broadly categorizing occupations to be, e.g., either cognitive or manual, this motivates a more detailed perception of occupations which focuses on the tasks performed. Model approaches that seek to identify the tasks which determine occupational growth are, therefore, needed to understand the observed changes in labor demand.
\\
This approach is being pursued by a line of literature that views occupations as bundles of tasks. For example, Firpo et al. (\citeyear{firpo2011occupational}) employ a five dimensional measurement for task contents of occupations which captures the effects of technological changes on occupational wages. In an application, this task content measure has explanatory value for changes in the distribution of occupational wages. Böhm (\citeyear{bohm2019price}) presents a propensity method with discrete task choice which can be utilized in the estimation of changes in task prices. Using this method, he identifies changes of task prices that are consistent with routine-biased technological change. Yamaguchi (\citeyear{yamaguchi2018changes}) builds on a task choice model that distinguishes between cognitive and motor tasks. He finds that differences in skill endowments between men and women, in combination with changes in returns to task specific skills that are driven by changes in technology, can explain changes in the gender wage gap.
\\
Adding to this line of literature, I present a model in which occupations are viewed as bundles of tasks. Within this theoretical framework, workers optimally choose from a continuity of task combinations under consideration of the resulting realized wage as well as the amenities associated with the choice of tasks. Following Böhm et al. (\citeyear{bohm2019occupation}), I rely on a distinction between realized wages and the wage paid per constant unit of skill (\textit{skill prices}) for the identification of skill specific changes in task prices which the authors argue to be direct reflections of shifts in labor demand. A worker's productivity in executing each of these tasks is determined by her endowment of the respective task specific skill.
\\
Through the identification of changes in relative task specific skill prices, this model can help to understand which tasks have experienced an increasing demand in the past and, thereby, what skills are increasingly being sought in the labor market. This contributes to the comprehension of differences in occupational growth.
\\
% TODO maybe delete the following paragraph: too technical for an introduction, reader does not know what to do with this information!
Utilizing a base period which is characterized by an absence of changes in relative skill prices, it is possible to identify changes in relative skill prices in subsequent periods in panel data. Building on the identification strategy derived in this thesis, I employ a simulation study to validate this finding. The results of a Monte Carlo estimation show that the method presented in this thesis can, indeed, be helpful for the identification of changes in relative skill prices.
\\ 
The remainder of this thesis is structured as follows. Section 2 introduces the model of continuous task choice. Beginning with a utility maximization problem, I present the identification strategy used to arrive at an estimable representation of changes in relative skill prices. Starting from a very general setting, I illustrate these findings by means of a model with two different tasks. Subsequently, section 3  presents an application of the identification strategy in the context of a simulation study. This proof of concept helps to examine previously established results in a Monte Carlo estimation of changes in relative task specific skill prices. A discussion and concluding remarks are provided in the final section 4 of this thesis.
\end{document} 