%!TEX root = ../main.tex
\documentclass[../main.tex]{subfiles}
\begin{document}

\section{A Model of Continuous Task Choice} \label{sec:tractable-model-of-cont-task-choice}
This section presents a model of continuous task choice which can be employed to analyze changes in task specific skill prices. In this framework, workers maximize utility by allocating  their working time between tasks. In doing so, workers are facing a trade-off between realized wages that are determined through their choice and associated amenities which depend on the workers preferences regarding their choice of tasks. This trade-off is reflected in a utility function defined as a composition of two parts: a pecuniary part, particularly realized wages, and a non-pecuniary part, i.e. the amenities associated with the executed tasks. The model I develop is related to the one presented in Böhm (\citeyear{bohm2019price}), where workers maximize their utility by choosing the task that offers the highest sum of pecuniary and non-pecuniary rewards. However, I deviate from this approach with discrete task choice in that I allow a continuous combination of tasks.
\\
As a starting point, I first introduce the utility maximization problem each worker faces in section \ref{sec:utility-max-problem}. In general, workers experience positive utility from wages which is accompanied by a dislike towards strong specialization in executed tasks as well as deviations from their preferred choice of tasks. In section \ref{sec:implications-with-j=2}, I present the solution of this choice problem by means of a two task version of this model. Subsequently, I derive the identification of changes in relative skill prices, based on the two task version of the model in section \ref{sec:identifying-changes-in-skill-prices}. Section \ref{sec:model_estimation} discusses how in an empirical application this identification strategy can be employed to estimate such changes in skill prices.

\subsection{The Utility Maximization Problem} \label{sec:utility-max-problem}
A worker's realized wage is modeled as a function of the task-specific wages that would be obtained if a worker spent her entire work-time on one particular task (\textit{potential wages}), and the fraction of work-time that she spends on each task, i.e. her \textit{task choice}.
\\
The amenity valuation associated with each task choice is determined by, first, an exogenous preference for a certain allocation of working time to tasks. In particular, agents experience negative utility from deviations from their preferred allocation. Second, amenities are modeled to capture a general dislike towards strong task-specialization. Notice that both these properties translate into non positive utility. With this conceptualization of the amenities at hand, it appears to be more intuitive to think of this non-pecuniary part of the utility function in terms of a penalty. Therefore, I will use the notion \textit{penalty term} in the further course of this thesis. 
% \footnote{(TODO: Motivate these modeling assumptions: (1) People have different likes for tasks; is there evidence for people sorting into different occupations/ sectors for given skills and wages? (3) People have a dislike towards very routine tasks; Is there evidence for people preferring less routine work?)}
\\
Since the objective of this analysis is an investigation of relative quantities, I use a notation in logarithmic values when describing my model. Above described properties are translated into a task choice model as follows. In each period $t \in \{0, ..., T\}$, all agents $i \in \{1, ..., N\}$ individually maximize their utility $u_{i,t}$ by choosing the fractions of their working time spent on each of $J$ tasks. These task choices are captured in $\lambda_{i,j,t}, \; \text{with} \; 0 \leq \lambda_{i,j,t} \leq 1, \; \text{and} \;j \in \{1, ..., J\}, \; \text{s.t.} \; \sum^J_{j=1} \lambda_{i,j,t} = 1$. This approach allows to represent occupations as continuous compositions of tasks.
\\ 
Realized wages of a worker result as the sum of potential task specific wages weighted by task choices. This definition of realized wage is captured in equation (\ref{eq:real_wage}), where $\boldsymbol{\lambda}_{i,t}$ and $\boldsymbol{w}_{i,t}$ denote vectors of length $J$ that contain the chosen work time fraction and the potential wage for each task, respectively. Alternatively, realized wages can be thought of as a linear combination of potential wages.
\begin{equation}\label{eq:real_wage}
	w(\boldsymbol{\lambda}_{i,t}, \boldsymbol{w}_{i,t}) = \sum^J_{j=1} \lambda_{i,j,t} w_{i,j,t}
\end{equation}
Following above description of negative amenities associated with task choices, the penalty term is defined as a function of task choices and preferred time allocations. Furthermore, this penalty incorporates the assumed dislike towards specialization. These properties are captured by the function presented in equation (\ref{eq:penalty}).
% TODO: take a different letter than capital Phi. 1. why capital? 2. cap Phi is Stammfunction
\begin{equation}\label{eq:penalty}
	\rho(\boldsymbol{b}_i, \boldsymbol{\lambda}_{i,t}, \phi, \theta) = \theta \sum^J_{j=1} |b_{i,j} - \lambda_{i,j,t}|^\phi
\end{equation}
In this functional form of the penalty term, $\boldsymbol{b}_{i,j}$ denotes a vector of length $J$ that captures the individually preferred time allocation on tasks, with $0 \leq b_{i,j} \leq 1 \;  \forall \; j \in \{1, ..., J\}$. This preference is assumed to be time invariant. From equation (\ref{eq:penalty}) one can see that the penalty is minimized at a value of zero by setting task choices exactly equal to the preferred allocation, i.e., $\lambda_{i,j,t} = b_{i,j} \; \forall \; j \in \{1, ..., J\}$. Ceteris paribus, increasing the difference between a worker's preferred time allocation and her respective task choice is associated with an increase of the penalty, hence decreasing utility.
\\
Parameter $\theta \in \mathbb{R}_+$ works as a scaling factor, allowing to adjust relative relevance of the amenity and wage terms for the utility function. This weighting parameter is limited to positive values since switching signs of this function results in a loss of its previously described desired properties. All else equal, an increase in penalty weight $\theta$ entails higher penalties from a deviation from the preferred time allocation. As seen from the absence of subscripts, $\theta$ is modeled to be constant across individuals and time. 
\\
Finally, the penalty term's exponent $\phi$ determines the dislike towards strong task specializations. Since the base of this exponent by construction is between zero and one (i.e., $|b_{i,j} - \lambda_{i,j,t}| \in [0,1]$), a larger exponent $\phi$, ceteris paribus, decreases penalties from deviations. $\phi$ is required to be weakly positive in order to ensure continuity of the first order condition of the resulting maximization problem (see section \ref{sec:implications-with-j=2} below). Similar to the penalty weight parameter, $\phi$ is assumed to be time invariant and identical across individuals as well as tasks.
\\
Equation (\ref{eq:utility}) below presents both components, realized wage and penalty term, combined to the utility function. Overall, the functional form of the penalty term is modeled to be time invariant, which will be important for the estimation strategy of changes in task prices in section \ref{sec:model_estimation}.
\begin{align}\label{eq:utility}
	u_{i,t} &= w(\boldsymbol{\lambda}_{i,t}, \boldsymbol{w}_{i,t}) - \rho(\boldsymbol{b}_i, \boldsymbol{\lambda}_{i,t}, \phi, \theta)  \\
	{}		&= \sum^J_{j=1} \lambda_{i,j,t} w_{i,j,t} - \theta \sum^J_{j=1} |b_{i,j} - \lambda_{i,j,t}|^\phi \nonumber
\end{align}
Each worker is assumed to act utility maximizing by setting task choices. Solving this optimization problem provides optimal task choices $\boldsymbol{\lambda}^*_{i, t}$, which are defined as follows.
\begin{equation} \label{eq:optimization_problem}
	\boldsymbol{\lambda}^*_{i, t} \equiv \argmax_{\boldsymbol{\lambda}_{i, t}} u_{i,t}(\boldsymbol{b}_i, \boldsymbol{\lambda}_{i, t}, \phi, \theta) = \argmax_{\boldsymbol{\lambda}_{i, t}} w(\boldsymbol{\lambda}_{i,t}, \boldsymbol{w}_{i,t}) - \rho(\boldsymbol{b}_i, \boldsymbol{\lambda}_{i,t}, \phi, \theta).
\end{equation}

\subsection{Implications in a Model with J=2 Tasks} \label{sec:implications-with-j=2}
Up to this point the model introduced in this thesis captures the general setting of a continuous choice between $J$ tasks. With the complexity of occupations in the real world in mind, a subdivision of each occupation into a multitude of tasks would be conceivable. The desired number of dimensions of this decomposition depends on the specific context of the model and can be specified accordingly. In this subsection, I consider a very tractable version of this model with $J=2$ tasks. This facilitates solving the model and results in accessible interpretations. The insights gained from this specification, however, are transferable to a higher dimensional task differentiation.
\\
A distinction between $J=2$ tasks could, for instance, be used to analyze a setting similar to the one presented in Yamaguchi (\citeyear{yamaguchi2018changes}) who distinguishes between "cognitive" and manual (specifically, "motor") tasks. Similarly, Beaudry et al. (\citeyear{beaudry2016great}) build their arguments on a model that considers two distinct tasks: "cognitive" and "routine". 
\\
A specification of the model with $J=2$ tasks allows to restate the utility function in equation (\ref{eq:utility}) in a more concise way, using that both, $\lambda_{i,j,t}$ and $b_{i,j}$ sum to one over all tasks. Furthermore, I normalize potential task specific wages, so that $w_{i,j=1,t} = 0$  in every period $t \in \{1, ..., T\}$. In this setting, potential wages in task two yield a quantity that is relative to the potential wage in task one. This does not impact the model's informative value regarding the relative task specific wages and will facilitate the identification thereof. Consequently, the utility function of a case with $J=2$ tasks results as following expression, the derivation of which can be found in appendix \ref{app:derive_two_task_utility}.
\begin{align}\label{eq:two_task_utility}
	u_{i,t} 		& = \lambda_{i,t} \tilde{w}_{i,t} - 2 \theta |b_i - \lambda_{i,t}|^\phi,  \\
	\text{where} 	& {} \nonumber \\
	\lambda_{i,t} 	& \equiv \lambda_{i,j=2,t}  \nonumber \\
	b_i 			& \equiv b_{i,j=2} \nonumber \\
	\tilde{w}_{i,t} & \equiv w_{i,j=2,t} - w_{i,j=1,t}. \nonumber
\end{align}
Drawing on the utility maximization problem, which is presented in a general version in equation (\ref{eq:optimization_problem}), workers maximize utility by optimally setting choice parameter $\lambda_{i,t}$. Thus, the first order condition presented in equation (\ref{eq:foc}) must be satisfied. This results in the piecewise defined optimal task choice presented in equation (\ref{eq:lmb_opt})\footnote{Note that $\tilde{w}_{i,t} < 0$, if and only if the potential wage in task $2$ is smaller then the potential wage in task $1$. This case is accompanied by a shift in task choices towards task $1$, which will result in $b_i > \lambda_{i,t}$ (this corresponds to the second case in equation (\ref{eq:lmb_opt})).  Analogously, it can be seen that $\tilde{w}_{i,t} > 0 \iff b_i < \lambda_{i,t}$. The exponent in equation (\ref{eq:lmb_opt}), therefore, is by construction always over a weakly positive base and thereby defined for any real number of $\tilde{w}_{i,t}$.}. For a derivation of the first and second order conditions of this optimization problem, see appendix \ref{app:foc_soc_utility}. A detailed derivation of optimal task choices $\lambda_{i,t}^*$ can be found in appendix \ref{app:derive_lmb_opt}. 
\begin{equation} \label{eq:foc}
	\frac{\partial u_{i,t}(b_i, \lambda_{i,t}, \phi, \theta)}{\partial \lambda_{i,t}} \overset{!}{=} 0
\end{equation}
\begin{equation} \label{eq:lmb_opt}
	\lambda^*_{i,t} = \left\{
	\begin{array}{ll}
		b_i - (\frac{- \tilde{w}_{i,t}}{2 \phi \theta})^{\frac{1}{\phi -1}}, \: & \: \text{if $b_i \geq \lambda_{i,t}$}\\
		b_i + (\frac{\tilde{w}_{i,t}}{2 \phi \theta})^{\frac{1}{\phi - 1}}, \: & \: \text{if $b_i < \lambda_{i,t}$}
	\end{array}
\right.
\end{equation}
From this representation of optimal task choices one can see that the penalty exponent $\phi$ is required to be weakly positive in order to provide continuity of optimal choices.\footnote{Considering equation (\ref{eq:penalty}), one can easily see that $\lim_{|b_i - \lambda_{i,t}|\to 0} \rho (\theta, \lambda_{i,t}, b_i, \phi) = \infty, \: \forall \phi < 0 \: \text{and} \: \theta > 0$} Apart from this, constraining $\phi > 1$ ensures strict concavity of the utility function which is the sufficient condition for a global maximum of $u_{i,t}$ at $\lambda_{i,t}^*$, given its domain of definition, $\lambda_{i,t}^* \in [0,1]$ (see appendix \ref{app:foc_soc_utility} for a derivation of this result). 
\\
Another way to understand the implications of choices of the penalty exponent $\phi$ on the solution to this maximization problem is in terms of the optimal task choice. Allowing for $0 \leq \phi \leq 1$ results in a weakly convex utility function with a kink point at $\lambda_{i,t} = b_i$. In this case, it can be seen from the first derivative of the utility function (appendix \ref{app:foc_soc_utility}) that workers will either stick with their preferred time allocation $b_i$ regardless of relative wages or changes thereof, or alternatively they will fully focus on one single task as soon as its relative wage is high enough, resulting in corner solutions. In such a model specification at most three optimal task choices are feasible: $\lambda_{i,t}^* \in \{0, b_i, 1\}$. While the first case results in a trivial version of the model without inter temporal task adjustments, the second case results in a model where each worker either sticks with her preferred task split $b_i$ or fully focuses on the higher paid task. Both cases are not desirable for the analyses in this thesis, so that it is throughout this work assumed that $\phi > 1$.
\\
Furthermore, notice that the solution presented above accommodates the special case of $\phi = 2$ in which, first, the penalty term no longer needs to be defined piecewise, and second, optimal task choices are linear in changes in relative potential wages. I will fall back to this case in the context of the simulation study that is presented in section \ref{sec:simulation_study} of this thesis.

\subsection{Identifying Changes in Skill Prices} \label{sec:identifying-changes-in-skill-prices}
Previous subsection presents the solution to the maximization problem workers are facing. Following this, I now show how the changes in skill prices over discrete time changes can be identified in this framework. Detailed derivations of the results presented throughout this section can be found in appendix \ref{app:ident_changes_in_skill_prices}. Starting from the utility function in the case of $J=2$ different tasks, presented in equation (\ref{eq:two_task_utility}), changes in time are captured in the total derivative thereof with respect to time. Using the product and chain rules, this yields the expression in equation (\ref{eq:util_total_derivative}).
\\
For the total derivative with respect to time, I make use of a more detailed notation where, instead of using subscripts to denote time dependence, such variables are written as explicit functions of time. Relative potential wages ($\tilde{w}_{i,t}$), thus are expressed as $\tilde{w}_{i}(t)$. Consequently, optimal task choices result as an indirect function of time, i.e. $\lambda_{i,t}^* \equiv \lambda_i^*(\tilde{w}_i (t))$, as seen from equation (\ref{eq:lmb_opt}). The utility function follows analogously.
\begin{align} \label{eq:util_total_derivative}
	\frac{d }{dt} u_{i}(\lambda_i^*(\tilde{w}_i(t)), \tilde{w}_i(t)) &=	\frac{\partial u_i(\lambda_i^*(\tilde{w}_i(t)), \tilde{w}_i(t))}{\partial \lambda_i^*(\tilde{w}_i(t))} \frac{\partial \lambda_i^*(\tilde{w}_i(t))}{\partial \tilde{w}_i(t)}  \frac{d \tilde{w}_i(t)}{dt} \nonumber \\
	{} &+ \frac{\partial u_i(\lambda_i^*(\tilde{w}_i(t)), \tilde{w}_i(t))} {\partial \tilde{w}_i(t)} \frac{d \tilde{w}_i}{dt} 
\end{align}
By application of the envelope theorem, the indirect effect of changes in relative potential wages via a change in optimal task choices $\lambda_i^*$ equates to zero so that only the direct effect of changes in relative wages needs to be considered.\footnote{As seen from equation (\ref{eq:foc}), the partial derivative of the utility function with respect to the task choice parameter must be zero by the first order condition of the utility maximization problem.} The total derivative of utility with respect to time, therefore, can be restated as follows.
\begin{equation} \label{eq:util_derivative_env_th}
	\frac{d }{dt} u_{i}(\lambda_i^*(\tilde{w}_i(t)), \tilde{w}_i(t)) = \lambda^*(\tilde{w}_i(t)) \frac{d \tilde{w}_i (t)}{d t}
\end{equation}
For the problem addressed in this thesis, however, changes in discrete time need to be identified. Therefore, equation (\ref{eq:util_derivative_env_th}) is integrated from $t-1$ to $t$. Throughout the remainder of this thesis, I use $\Delta$ as notation for discrete changes of time-dependent variables. Hence, I formulate the integral over marginal changes in equation (\ref{eq:util_derivative_env_th}) from $t-1$ to $t$ as follows. 
\begin{equation} \label{eq:util_integration}
	\Delta u_{i,t} = \int^t_{t-1} \left( \frac{d}{d \tau} u_i\big(\lambda_i^*(\tilde{w}_i(\tau))\big) \right) d\tau = \int^{\tilde{w}_{i,t}}_{\tilde{w}_{i,t-1}} \lambda_i^*(\tilde{w}_{i, \tau}) d\tilde{w}_{i, \tau}, 
\end{equation}
where in the last part of expression (\ref{eq:util_integration}) integration by substitution is used to restate the integral in the second term. By substituting $\tilde{w}_t(\tau)$ with $\tilde{w}_{i,\tau}$, I again rely on the abbreviated notation of time dependence, which I will stick with for the remainder of this work. 
\\
Applying a linear interpolation of task choices, presented in equation (\ref{eq:lambda_approximation}), leads to the intermediate result which is shown in equation (\ref{eq:utility_change}). For the identification of the model, this approximation of $\lambda_{i,\tau}^*$ allows to arrive at discrete changes in utility as a function of discrete changes in potential wages. However, against the background of an empirical application of this result, a different interpretation is conceivable: For an unknown functional form of optimal task choices, a linear interpolation between two observable points in time offers a approximation of the interim task choices.
\begin{equation} \label{eq:lambda_approximation}
	\lambda^*_i(\tilde{w}_{i,\tau}) \approx \lambda^* + \frac{\lambda^*(\tilde{w}_{i,t}) - \lambda^*(\tilde{w}_{i,t-1})}{\tilde{w}_{i,t} - \tilde{w}_{i,t-1}} (\tilde{w}_{i,\tau} - \tilde{w}_{i,t-1})
\end{equation}
\begin{equation} \label{eq:utility_change}
	\Delta u_{i,t} = \bar{\lambda}_{i,t}^* \Delta \tilde{w}_{i,t}
\end{equation}
Where $\bar{\lambda}_{i,t}^* \equiv \frac{\lambda_{i,t-1}^* + \lambda_{i,t}^*}{2}$ denotes the mean task choice of periods $t-1$ and $t$. Notice that the linear interpolation of optimal task choices is not an approximation in case the penalty function is quadratic (i.e., penalty exponent $\phi = 2$). It can be seen from equation (\ref{eq:lmb_opt}) that a quadratic penalty term will result in optimal task choices $\lambda_{i,t}^*$ that are linear in relative potential wages. The functional form of optimal task choices is not an approximation but exactly correct in this case.
\\ 
Similar to the model presented in Böhm et al. (\citeyear{bohm2019occupation}), the key identifying assumption is the divisibility of potential wages into wage paid per constant unit of skill, i.e. skill prices ($\pi_{j,t}$), and skill endowments ($s_{i,j,t}$) In log notation, this assumption allows for a decomposition as shown in equation (\ref{eq:potential_wages}) for each task $j \in \{1, ...,J\}$.
\begin{equation} \label{eq:potential_wages}
	w_{i,j,t} = \pi_{j,t} + s_{i,j,t} 
\end{equation}
This implies that changes in relative potential wages can be decomposed into changes in relative skill endowments and changes in relative skill prices, i.e. $\Delta \tilde{w}_{i,t} = \Delta \tilde{\pi}_i + \Delta \tilde{s}_{i,t}$. Similarly, changes in utility can be split into wage changes and changes in the penalty term, as shown in the general version of the utility function presented in equation (\ref{eq:utility}), i.e. $\Delta u_{i,t} = \Delta w_{i,t} - \Delta \rho_{i,t}$. Making use of both decompositions, equation (\ref{eq:utility_change}) can be restated to an expression of wage changes:
\begin{equation} \label{eq:wage_change}
 	\Delta w_{i,t} = \bar{\lambda}^*_{i,t} \Delta \tilde{\pi}_{t} + \bar{\lambda}^*_{i,t} \Delta \tilde{s}_{i,t} + \left[ \rho_{i}(\lambda_{i,t}^*) - \rho_{i}(\lambda_{i,t-1}^*) \right]
\end{equation} 
Considering equation (\ref{eq:wage_change}), changes in observable realized wages can be partitioned into three subcomponents: First, changes in skill prices ($\bar{\lambda}^*_{i,t} \Delta \tilde{\pi}_{t}$), which is what this section aims to identify. Second, changes in skill endowments ($\bar{\lambda}^*_{i,t} \Delta \tilde{s}_{i,t}$) that result from some skill accumulation process, and third, changes in the penalty term ($\rho_{i}(\lambda_{i,t}^*) - \rho_{i}(\lambda_{i,t-1}^*)$). Identifying the relation between changes in realized wages and changes in skill prices, therefore, requires the determining both, the second and third summands of equation (\ref{eq:wage_change}) in a first step. In the following section I hence discuss the strategy used to distinguish price changes from these other influences on realized wages.

\subsection{Estimation of the Model} \label{sec:model_estimation}
Equation (\ref{eq:wage_change}) shows that determining changes in relative skill prices requires to distinguish them from changes in both, skill endowments as well as in the penalty term. In an estimation with optimal task choices as explanatory variables price changes would not be distinguishable from changes in skill endowments. Moreover, it would require controlling for changes in the penalty term. A method to isolate changes in skill prices is presented in Böhm et al. (\citeyear{bohm2019occupation}). In a setup that is simiar to the framework presented in this thesis, the authors distinguish between effects of skill accumulation and changes in skill prices relying on a base period $t=0, ...., T_{base}$ in which they set $\Delta \pi_t = 0$. In the article, the authors argue that this can either be seen as an assumption or as a normalization. In the context of this thesis, it is straight forward to think of it as a normalization for two reasons. First, the model introduced in this work by construction cannot provide insights on absolute levels of prices, but relative prices instead. It is, therefore, intuitive to interpret changes in skill prices relative to a base period. Second, in order to answer the research question of this thesis on changes in the relative importance of different task specific skills, one inherently has to consider changes in task prices in comparison to some earlier period.
\\
In addition to skill accumulation, the base period can also be used to estimate changes in the penalty term. In this model, the penalty term is a direct function of optimal task choices, only. In the absence of changes in skill prices, it is therefore possible to estimate $\Delta \rho_{i,t} = \rho_i(\lambda_{i,t}^*) - \rho_i(\lambda_{i,t-1}^*)$ and treat it as a confounding variable in subsequent periods.\footnote{See Böhm et al. \citeyear{bohm2019occupation} who present similar, yet more general results on a model with "non-pecuniary benefits" with unknown functional form.} Exploiting the base period to estimate both, skill accumulation and changes in the penalty term, comes with the downside that these estimated effects cannot be distinguished. However, this is not of relevance to the specific question of this thesis, which is focused on the identification of skill prices. Furthermore, there are other, more sophisticated approaches to work out the skill accumulation process in panel data. These I will discuss in greater detail in section \ref{sec:conclusions}.
\\
The estimation results for the contributions of skill accumulation and changes in the penalty term can be used to adjust wage changes in subsequent periods for these effects. Particularly, the assumption that both, the skill accumulation process as well the functional form of the penalty term are time invariant allows to transfer the estimated results from the base period to the subsequent period. Finally, an estimation of changes in skill prices can be obtained by regressing the residual wage change, i.e. observed change in realized wages adjusted for changes in skill endowments and penalty term, on average task choices.
\end{document}