%!TEX root = ../main.tex

\documentclass[../main.tex]{subfiles}
\begin{document}

\section{Conclusions}
\label{sec:conclusions}
% What do I do in this thesis? 1-3 sentences. Top level description
This thesis presents a method to estimate changes in task specific skill prices. In a theoretical framework of continuous task choice, workers behave utility maximizing by optimally choosing in what tasks they engage under consideration of resulting wages and accompanying occupational amenities. The identification strategy presented in this thesis can be used to infer on changes in relative skill prices from observable wage changes and task choices in panel data. As a proof of concept, I test this theoretical result in a simulation study. Herein, simulated changes in skill prices are estimated with a slight bias. Depending on the parameterization of the estimation of changes in skill accumulation and the penalty term, changes in relative skill prices are either underestimated by up to $2.5 \%$ or overestimated by up to $5 \%$.
\\
% Next step: apply this method to real world panel data. 
%% Here it could be useful to estimate skill accumulation as well as a broader model of amenities in a more saturated model of personal traits (like Böhm et al propose in their appendix)
After discussing the model in a theoretical framework and testing its implications in a simulation study, the next step should be the application of the method that is developed in this thesis on actual panel data. This could provide valuable insights into changes in labor demand at a task level. These insights could help to better understand the larger shifts in labor demand that are summarized in the concept of job polarization. Such an application to panel data would, furthermore, provide the opportunity to refine the estimation method. In section \ref{sec:identifying-changes-in-skill-prices} I show that wage changes result as a composition of changes in task specific skill prices as well as skill accumulation and changes in amenities. The estimation of both, skill accumulation as well as changes in amenities could gain in precision using workers' observable characteristics.
\\
Particularly, Böhm et al. (\citeyear{bohm2019occupation}) make use of a saturated skill model to estimate changes in skills as a result of occupational choices and observable characteristics controlling the speed of skill acquisition of a worker. In their online appendix, Böhm et al. (\citeyear{bohm2019occupation}), furthermore, provide a method to estimate changes in the relative amenity value of occupational choices between periods and in comparison to a reference occupation. In a setting where the personal characteristics of workers allow an indication of the amenities experienced from their task choice, the authors present a method that enables an estimation of changes in amenities using workers' task choice and characteristics.
\\
As discussed in section 3.2, the method for identifying changes in skills as well as the penalty term leads to a slight over-, or respectively underestimation of changes in relative skill prices. With help of the more sophisticated estimation method for these effects, discussed in Böhm et al. \citeyear{bohm2019occupation}, the results of an estimation of changes in relative skill prices in the theoretical framework of the model presented in this thesis could be improved. However, since this method relies on information on individual characteristics of workers, it requires the inclusion of an extensive panel data set.
\\
With the model of continuous task choice that is presented in this thesis at hand, an arbitrarily differentiated examination of changes in relevance of different occupational tasks for labor market outcomes in the sense of employment and wage growth is possible. In contrast to the discrete division into routine versus non-routine or cognitive versus manual tasks that is usually the basis in similar studies, this model allows for a continuous scale. This can potentially increase the precision of estimations of task prices by attributing wages to occupational tasks more accurately. Additionally, this framework allows to consider a larger number of occupational tasks and according skills in a comprehensive model.
% As part of this application on real panel data, discuss methods to identify tasks
%% "time series observations on the task composition of jobs from the dictionary of occupational titles" Autor 2003 
%% Paper Rohrbach-schmidt, tiemann provides discussion on task data in germany and provides instructions for creating task-indices
%% Deming presents a "skill intensity measure" based on O*NET data
% Identifying changes in skill prices could be helpful in understanding larger shifts in labor demand - similar to what e.g., Deming presents: What tasks experienced growing demand throughout recent years
% Policy implications: only by identifying what skills are favourable for economic outcomes (employablility and wage), they can be appropriately promoted
% One shortcoming of the model presented in this thesis is the neglected complementarity of tasks. The way wages are understood in this model does not allow for interactions in the valuation of task skills. 
\end{document}