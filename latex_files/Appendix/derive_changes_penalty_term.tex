%!TEX root = ../main.tex

\documentclass[../main.tex]{subfiles}
\begin{document}
\subsection{Derivation of Changes in the Penalty Term}\label{app:derive_changes_penalty_term}
% Used in section 3 - MC Simulation
As part of the estimation of changes in skill prices, changes in the penalty term need to be estimated. These changes in the penalty term are a funtion of optimal task choices $\lambda_{i,\tau}^*, \; \tau \in \{t-1, t\}$. In this appendix, I show that changes in the penalty term result as a polynomial with penalty exponent of degree $\phi$.\\
From equation (\ref{eq:wage_change}) we have that wage changes come as a composition of three parts:
\begin{equation}
	\Delta w_{i,t} = \bar{\lambda}^*_{i,t} \Delta \tilde{\pi}_{t} + \bar{\lambda}^*_{i,t} \Delta \tilde{s}_{i,t} + \left[ \rho_{i}(\lambda_{i,t}^*) - \rho_{i}(\lambda_{i,t-1}^*) \right] \tag{\ref{eq:wage_change}}
\end{equation} 
In the following I show that the part related to changes in the penalty term (i.e., $\left[ \rho_{i}(\lambda_{i,t}^*) - \rho_{i}(\lambda_{i,t-1}^*) \right]$) can be restated as a polynomial of optimal task choice parameters $\lambda_{i,\tau}^*, \; \tau \in \{t-1, t\}$ of degree $\phi$.
\begin{align}
	\intertext{Starting from}
	{} &  \rho_{i}(\lambda_{i,t}^*) - \rho_{i}(\lambda_{i,t-1}^*)  \label{eq:appen_penalty_term_1} \\
	\intertext{with}
	\rho_{i}(\lambda_{i,\tau}^*) &= \theta |b_i - \lambda_{i,\tau}^*|^\phi, \; \text{for} \; \tau \in \{t-1, t \} \nonumber \\
	\intertext{With regard to the absolute value function, two cases have to be considered:}
		|b_i - \lambda_{i,\tau}^*| &= \left\{
		\begin{array}{ll}
			b_i - \lambda_{i, \tau}^*, &\text{if} \; b_i - \lambda_{i, \tau}^* \geq 0 \\
			-(b_i - \lambda_{i, \tau}^*), &\text{if} \; b_i - \lambda_{i, \tau}^* < 0
		\end{array} 
	\right. \nonumber \\
	\intertext{Consider first case: $b_i - \lambda_{i, \tau}^* \geq 0$. In this case equation \ref{eq:appen_penalty_term_1} can be written as follows:}
	\rho_{i}(\lambda_{i,t}^*) - \rho_{i}(\lambda_{i,t-1}^*) &= (b_i - \lambda_{i,t}^*)^\phi - (b_i - \lambda_{i,t-1}^*)^\phi \nonumber
	\intertext{Each of the two binomials on the right hand side can be rewritten as bivariate polynomials by application of the binomial expansion:}
	(b_i - \lambda_{i,\tau}^*)^\phi &= \sum^\phi_{k=0}\binom{\phi}{k} b_i^{\phi-k} (-\lambda_{i,\tau})^k, \; \text{for} \; \tau \in \{t-1, t\} \nonumber
\end{align}
It is straightforward to see that in the second case ($ b_i - \lambda_{i, \tau}^* < 0$) the result in analogous with opposite sign.\\
Thus, changes in the penalty term can be expressed as the differencee between two polynomials of $\lambda_{i,t-1}$ and $\lambda_{i,t}$, respectively. The degree of the resulting polynomial is equal to the penalty exponent $\phi$.\\
In the simulation study, it is assumed that $\phi = 2$. The changes in the penalty function in this particular case, therefore, result as follows:
\begin{align}
	\rho_{i}(\lambda_{i,t}^*) - \rho_{i}(\lambda_{i,t-1}^*) &= (b_i - \lambda_{i,t})^2 - (b_i - \lambda_{i,t-1})^2 \nonumber \\
	{} &= \lambda_{i,t-1}^{*2} + \lambda_{i,t}^{*2} + 2b_i(\lambda_{i,t-1}^* + \lambda_{i,t}^*) \tag{\ref{eq:changes_in_penalty}}
\end{align}

\end{document}